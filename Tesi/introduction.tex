\section{Food problems and their impact on childhood }
One of the main problems of modern society is the one related to nutrition. Nutrition has changed a lot in history thanks to the industrial development that allowed the massive production of foods that in the past were produced only by hand (and with high costs) and thanks also to scientific progresses that allowed the discovery of food conservation and their elaboration to obtain new kinds of food that are more suitable to our need. Montignac \cite{Lastoriadell'alimentazionedell'uomo.} also identifies other causes that brings the concept of "nutrition" to assume the current meaning, for example the habits evolution and the female emancipation that has changed the ancient vision of women as "landladies" and that has promoted the progression of the "ready meals" industry and therefore of pre-cocked and packaged meals that today are consumed increasingly. However, the main phenomenon that has taken place in our era is the one related to the globalization and standardization of destabilized north american eating habits that has promoted the global growth of fast-food, indicated by WHO (World Health Organization) as a "pandemic" since 1997 given its extraordinary expansion that carried also a lot of other problems.\\
Childhood obesity is surely one of the clearest examples of these diet's changes of the new millennium.
According to a seminar held by CB Ebbeling, DB Pawlak and DS Ludwig \cite{Childhoodobesity} childhood obesity is a phenomenon that has had a great increase in all the world in the last twenty years, as we can see from this diagram.\clearpage
\begin{figure}[H]
\centering
\includegraphics[width=13cm, height=10cm]{immagini/obesity.png}
\caption{Childhood obesity diagram}\label{fig:obesity}
\end{figure}
Historically, a fat child was seen as healthy because he was likely to survive better to illnesses and infections; however, excessive fatness has become one of the most diffused health problem in children. The three experts have underlined how the problem is most common in developed and industrialized nations in which diet has changed radically favouring foods containing saturated and trans fat and with high glycaemic index, typical of fast-food in which also bigger portions are served. Moreover, this foods are also poor of fibre, micronutrients and antioxidants that the body needs for a correct functioning of metabolism. The excessive consuming of these foods brings the child to have health problems such as heart diseases, vascular disorders, hypertension, chronic infiammations and diabete of type 2, illness that in the past was not present in teenagers, but that now has had a rapid spread.\linebreak 

Another important problem that affects the food safety of children is represented by allergies.
According to the data collected by AM Branum and SL Lukacs \cite{FoodallergyUSchildren} it is possible to observe an increasing in cases of all kinds of food allergies including milk, eggs, peanuts, tree nuts, fish, shellfish, soy and wheat of around the 18 per cent on individuals under the age of 18 from the 1997 onwards in US (but we have reasons to believe that this can also be found in all the industrialized world). Reactions to these foods may vary from small diseases to anaphylactic shock that, in severe cases, could lead to death.
The researches have also underlined how, in the same period analized previously, there was also an increasing of hospital discharges (clearly after an hospitalisation) due to allergic reactions as we can see from this barplot.
\begin{figure}[H]
\centering
\includegraphics[width=10cm, height=7cm]{immagini/allergybarplot.png}
\caption{Allergy hospital discharges barplot}\label{fig:allergybarplot}
\end{figure}
All these problems that have been reinforced in recent years, lead us to think that it is necessary to support nutrition education and to make it a fundamental thing during childhood and adolescence, in order to empower everyone to a correct care of their health.

\section{Virtual Reality}
\section{NDD}
\section{GEA}
Descrizione ed evoluzione di GEA
\subsection{Thesi Structure}
The thesis is organized as follows:
\begin{description}
\item[Chapter 2 (State of the art)] In this section we show all the technologies for Virtual Reality, explaining how they works and their relation with NDD people. We present also an instrument very useful for terapist that allow to replicate the smartphone's screen, Google Chromecast, and the touchscreen evolution. Finally we describe the projects already developed about food education.
\item[Chapter 3 (Target groups, needs and requirements)] Here we present the onlus we collaborate with to better understand the NDD problematic and to evaluate our application. Next we talk about our context, need and goals, all the requirements we have and all the constraints are imposed to us.
\item[Chapter 4 (Design)] In this chapter we put an high-level description of our project and every application's part is described in details. After that we put possible scenarios and the user experience. 
\item[Chapter 5 (Implementation)] We present all the tools used to build up our application and the hardaware and software architecture description.
\item[Chapter 6 (Content issues)]POSSIBILI PROBLEMATICHE SE PRESENTI
\item[Chapter 7 (Evaluation)] We show the evaluation sessions describing all the results, feedbacks and opinions we collect to improve our project.
\item[Chapter 8 (Value proposition)] We discuss about the effects that our application has on the target group but also on therapists and families. PAPER SE CE LA FAMO.
\item[Chapter 9 (Future work)] Possible future projects that comes from this one.
\end{description}
 
\subsection{Origin of the name}
We decide to give the name GEA to our application because of two reasons. The first one is that Gea, in the greek mythology, was the personification of the Earth, the mother of all life so she is also the symbol of the nature that recalls the nutrition's topic. The second motivation is that in Italian Gea is the acronym of "Gioco Educazione Alimentare", which translated is "Food Education Game", in this way in the title there is the objective's explanation.\\
LOGOOOOO

