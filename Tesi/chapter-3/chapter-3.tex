These are very personal suggestions, no offense will be taken if you
completely ignore this chapter.

\section{Manage bibliography}

Manage a lot of bibliography resources by manually editing a \textsf{bibtex}
file is very annoying, it is better to use something that manages
for you all the references. Sadly, \LyX{} does not have an easy to
use system for doing that. However, there are other programs that
can be used to automagically organize dozens of papers.

There are two famous programs\textsf{, referencer} and \textsf{JabRef}
(\texttt{\url{https://launchpad.net/referencer},} \url{http://jabref.sourceforge.net/}),
both have a lot of useful features, the most notable are
\begin{itemize}
\item copy from clipboard of \textsf{bibtex} formatted references
\item organization of references with labels
\item possibility to associate a \textsf{pdf} to each reference
\item the \textsf{Cite in \LyX{} }button
\end{itemize}

\section{Reviews}

Your advisor will review what are you writing and he will, most likely,
add annotations on a \textsf{pdf}. However, correction contained inside
annotations are difficult to integrate, manly because you will spend
a lot of time in finding where to modify your document. \LyX{} supports
a very powerful method to review a document, just look the examples
below.

Where have you learned english stupid dumbass?

The derivative of $e^{x}$ is $e^{2x}$ .

After having considered all the other solutions we proved that this
is the most efficient way to determine the medium length of horse's
mane.

These result shows that the first method is way better than the second.
\\
The integration of these reviews is much easier than reviews inside
annotation of a \textsf{pdf}. Unfortunately you have to convince your
advisor to use this system, I think it is worth a try.

This feature is called \textsf{Change Tracking}, there is a dedicated
toolbar that you can show by activating the \textsf{Change Track}
option (\textsf{Document \lyxarrow{} Change Tracking \lyxarrow{} Track
Changes} or simply \textsf{CTRL + SHIFT + E}). This document has the
\textsf{Change Tracking} option already enabled.

\section{Final presentation}

Once you have written your thesis you will have to present it. Timing
is critical. You will have from $15$ to $20$ minutes to present
the work of months. A very useful tool that you may use during your
presentation is \texttt{pdfpc} (\url{http://davvil.github.io/pdfpc/}).
By using this program, on the screen of your laptop you will have
some additional information that are not showed on the external monitor.
These extra information are
\begin{description}
\item [{Time~left}] you can set the duration of the presentation and see
how many minutes and seconds you have left
\item [{Next~slide}] you will see both the current and the next slide
\item [{Annotations}] you can add some annotation to remember you some
key points that you may forget
\end{description}
This tool is very very useful, but do not forget that there is always
the demo effect. You MUST try it before even thinking to use for your
final presentation.

To install the program on a Linux system there should be a package
named \textsf{pdf-presenter-console}, for other OS on the websites
of the program there are the instruction how to install it.
