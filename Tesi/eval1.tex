In the first evaluation, we introduced GEA to a group of therapists and a group of patients from Fraternit\'a \& Amicizia (one of the therapeutic centers that collaborated with us in the requirements elicitation phase) and we made a first assessment of the tool's usability from their point of view, collecting useful suggestions and comments. This was done with the collaboration of "TWB" project (Therapeutic services based on Wearable virtual reality and Bio-sensors) better explained in \cite{TWB}.\\
For privacy issues, together with the centre we collaborate with, we established to use aliases instead of real names.
\section{Therapists}
The therapists come from the staff of Fraternit\'a \& Amicizia and include psychologists and experts of NDD working daily with this class of subjects. The location of the evaluation was "Fraternit\'a \& Amicizia" - Soc. Cooperativa Sociale - Onlus - Via Washington 59, Milan. The project involves 7 sessions, through a 2-month period (from Monday 8 September to Monday 19 November 2018), but only two of them were related with GEA so we write here only data related with respectively session number 2 and session number 3 and their results. In all the sessions each therapist practiced with TWB platform and services for about 20 minutes.\\
\begin{table}[H]
\centering
\tabulinesep=1.2mm
\newcolumntype{C}{>{\centering\arraybackslash} m{8cm} }
\newcolumntype{A}{>{\centering\arraybackslash} m{1.5cm} }
\newcolumntype{B}{>{\centering\arraybackslash} m{4.5cm} }
\begin{tabu} {|B|A|C|}
\hline
       \textbf{ALIAS} & \textbf{AGE} & \textbf{QUALIFICATION}\\\hline
       Operator One & BOH & BOH\\\hline
       Operator Two & BOH & BOH\\\hline
     Operator Three & BOH & BOH\\\hline
     Operator Four & BOH & BOH\\\hline
     Operator Five & BOH & BOH\\\hline
     Operator Six & BOH & BOH\\\hline
     Operator Seven & BOH & BOH\\\hline
     Operator Eight & BOH & BOH\\\hline
     Operator Nine & BOH & BOH\\\hline
     Operator Ten & BOH & BOH\\\hline
     Operator Eleven & BOH & BOH\\\hline
     Operator Twelve & BOH & BOH\\\hline
     Operator Thirteen & BOH & BOH\\\hline
     Operator Fourteen & BOH & BOH\\\hline
     Operator Fifteen & BOH & BOH\\\hline
     Operator Sixteen & BOH & BOH\\\hline
     Operator Seventeen & BOH & BOH\\\hline
     Operator Eighteen & BOH & BOH\\\hline
     Operator Nineteen & BOH & BOH\\\hline
     Operator Twenty & BOH & BOH\\\hline
     Operator Twenty-one & BOH & BOH\\\hline
     Operator Twenty-two & BOH & BOH\\\hline
     Operator Twenty-three & BOH & BOH\\\hline
     Operator Twenty-four & BOH & BOH\\\hline
     Operator Twenty-five & BOH & BOH\\\hline     
 \end{tabu}
 \caption{Therapists' data \label{table:datat}}
\end{table}
\subsection{Objectives}
\begin{itemize}
\item Usability
\item Adoptability
\item Therapeutic potential
\item Perceived utility
\item Collect feedback about how to improve the user interface
\item Collect feedback about possible extensions of the tool's functionalities
\end{itemize}
\subsection{Participants}
The evaluation was done with 25 therapists.
\subsection{Test setup}
The evaluation sessions with therapists where performed in 3 different rooms of Fraternit\'a \& Amicizia center in Milan, Via Washington 59. All the rooms were equipped with 1 PC running TWB web Platform, 1 headset and 1 mobile phone to play the WIVR application and, depending on the session, it was possible to have the EEG headset and Empatica Wristband. The room was also equipped with 1 videocamera with an integrated microphone to record each session. In each session, one expert of TWB and one therapist followed the entire session.
\subsection{Introduction to the test}
The week before the first session, a preliminary presentation of the entire TWB project and the evaluation process was organized with therapists, patients and patients' parents. During this meeting, we presented our system to them and explained how we wanted to perform the following experimental sessions.
\subsection{Test procedure}
We wrote down notes about reactions and comments coming from therapists while playing with the VR activities and after every session therapists had to fill in questionnaires to evaluate VR activities used during the specific session in terms of VR effectiveness, satisfaction, willingness to play again and engagement, both for them and for their patients. Moreover, questions about VR adoptability and perceived utility were asked and each session was recorded through a videocamera and a microphone to allow successive analysis.\\
During session number 2, performed on 15 October 2018, they play with the mini-game "Learn with the pyramid!".\\
During session number 3, performed on 22 Octber 2018, they play with the mini-games "Let's eat!" and "Healty or not?".
\subsection{Feedback}
\subsection{Results}
\subsection{Conclusions}

\section{Patients}
The patients, between age of 17 and 51 years old, were all selected among the users of Fraternit\'a \& Amicizia Cooperativa Sociale, the therapeutic center in Milan collaborating with us. The subjects differ in diagnosis and functioning level, as the following table shows. The project involve 7 sessions and the locations are:
\begin{itemize}
\item Fraternit\'a \& Amicizia - Soc. Cooperativa Sociale - Onlus - Via Foppa 7, Milano
\item Fraternit\'a \& Amicizia - Soc. Cooperativa Sociale - Onlus - Via Washington 59, Milano
\end{itemize} 
\begin{table}[H]
\centering
\tabulinesep=1.2mm
\newcolumntype{C}{>{\centering\arraybackslash} m{8cm} }
\newcolumntype{A}{>{\centering\arraybackslash} m{1.5cm} }
\newcolumntype{B}{>{\centering\arraybackslash} m{4.5cm} }
\begin{tabu} {|B|A|C|}
\hline
       \textbf{ALIAS} & \textbf{AGE} & \textbf{DIAGNOSIS}\\\hline
       User One & 27 & BOH\\\hline
       User Two & 26 & BOH\\\hline
     User Three & 19 & BOH\\\hline
     User Four & 26 & BOH\\\hline
     User Five & 28 & BOH\\\hline
     User Six & 32 & BOH\\\hline
     User Seven & 33 & BOH\\\hline
     User Eight & 34 & BOH\\\hline
     User Nine & 20 & BOH\\\hline
     User Ten & 17 & BOH\\\hline
     User Eleven & 37 & BOH\\\hline
     User Twelve & 32 & BOH\\\hline
     User Thirteen & 25 & BOH\\\hline
     User Fourteen & 20 & BOH\\\hline
     User Fifteen & 18 & BOH\\\hline
     User Sixteen & 31 & BOH\\\hline
     User Seventeen & 24 & BOH\\\hline
     User Eighteen & 27 & BOH\\\hline
     User Nineteen & 46 & BOH\\\hline
     User Twenty & 51 & BOH\\\hline    
 \end{tabu}
 \caption{Patients' data \label{table:datap}}
\end{table}
\subsection{Objectives}
\begin{itemize}
\item Biosensors acceptability and wearability
\item Virtual Reality usability and acceptability
\item Effectiveness variables: satisfaction, engagement, willingness
\item Effectiveness of WIVR applications in terms of skills improvement (attention, behavioral skills etc...)
\end{itemize}	
\subsection{Participants}
The evaluation was done with 20 patients.
\subsection{Test setup}
Evaluation sessions with patients where performed in 2 different centers belonging to Fraternit\'a \& Amicizia located in Milan: Via Washington 59 and Via Foppa 7. In both the centers, the rooms used for the evaluation were equipped with 1 PC running TWB web Platform, 1 headset and 1 mobile phone to play the WIVR applications and, depending on the session, it was possible to have the EEG headset and/or Empatica Wristband. \\
The room was also equipped with 1 videocamera with an integrated microphone to record each session. 
In each session, two experts of TWB and one therapist followed the entire session
\subsection{Introduction to the test}
The week before the first session, a preliminary presentation of the entire TWB project and the evaluation process was organized with therapists, patients and patients' parents. During this meeting, we presented our system to them and explained how we wanted to perform the following experimental sessions.
\subsection{Test procedure}
For the study the 20 patients are divided into 2 groups of 10: the first group performed the evaluation every Monday from 1st October to 12th November 2018, the other one met every Wednesday from 10th October to 28st November 2018 (except for Wednesday 1st November that is non-working day). The evaluation consisted of 7 sessions per patient, one per week (on Monday or Wednesday), lasting about 15-20 minutes, through a 2-month period (from 1st October to 21st November), during which patients experimented only the mini-game "Healthy or Not?". These activities (our "Healthy or not?" and another prototype) were chosen together with therapists and experts, among 25 possible activities, because they are considered the best ones to evaluate skills improvement. The supervisors were 2 TWB IT experts and 1 educator.\\
During session number 1 each patient tried to wear only Empatica Wristband, Empatica Wristband together with Emotiv EEG Headset, Empatica Writstband, Emotiv EEG Headset, Virtual Reality Headset. From S2 to S7 in the second half of each session, patient had to complete one level of the activity called "Healthy or Not?".\\
If during the first session the patient accepted to wear Empatica Wristband or Emotiv EEG Headset or both, he/she could wear them during these sessions in VR. 
During the execution of this study, we collected two types of data:
\begin{itemize}
\item Quantitative data, including both interaction data from the WIVR activities and physiological signals measured by the biosensors worn by patients:
\begin{itemize}
\item Interaction data: level of attention, number of errors, activity completion time
\item Data coming from Empatica: HR, BVP, EDA, Temperature
\item Data coming from Emotiv EEG: Excitement (Arousal), Interest (Valence), Stress (Frustration), Engagement/Boredom, Attention (Focus) and Meditation (Relaxation)
\end{itemize}	
\item Qualitative data, writing down errors, difficulties found and comments from the users in specific forms. Moreover, at the end of sessions 1, 2 and 7 users must answer to a questionnaire about what they experienced that day. Specifically:
\begin{itemize}
\item In session S1, we wrote down notes about reactions and comments coming from patients while wearing only Empatica, Empatica with EEG Headset and Empatica with EEG and VR Headset. After the session, each patient had to complete a questionnaire about biosensors acceptability and wearability.
\item From session S2 to S7, we used ad-hoc forms to collect information about errors and completion time for each task in VR. These data were used to measure VR activities effectiveness and patients' skills improvement.
\item After session S2 and S7, patients had to fill in questionnaires to evaluate VR acceptability, VR effectiveness as well as VR satisfaction, willingness to play again and engagement. We decided, together with experts, to give questionnaire to fill in only after first and last session in which patients had to use virtual reality: in this way we could see if some answers changed from the beginning to the end of the experimentation.
\item From session S2 to S7 it was possible to collect data coming from biosensors (Empatica Wristband or Emotiv EEG Headset or both), if patients accepted to wear them during their first session (S1).
\item Moreover, each session was recorded through a videocamera and a microphone to be revised later.
\end{itemize}
\end{itemize}

\subsection{Feedback}
\subsection{Results}
\subsection{Conclusions}
