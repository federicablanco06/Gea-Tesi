\section{Requirements elicitation}
The requirements, needs and goals for GEA were collected through meetings with psychologists and experts in the field of NDD, mainly Eleonora Beccaluva of the \textit{Fraternit\'a \& Amicizia Onlus} in Milan with whom we collaborated starting from the general idea up to the actual development. To better contextualize and define the themes of the educational game, we took part in some food laboratory activities organized in the aforementioned therapeutic center with patients with high functioning NDD. During these days the boys themselves showed a high interest in wanting to learn properly nutrition emphasizing a need for self-sufficiency that could be achieved through a game of support and continuation of educational activities usable by home and not only during dedicated hours. They showed us the environments in which they teach lessons explaining how they are structured, what are the focuses on which they focused and have specifically asked us to turn all this into a fun and educational game at the same time. Both patients and educators have shown themselves to be very supportive of the idea of using Virtual Reality for many positive factors such as being able to recreate the safe environment in which they are accustomed to work, thus maintaining the same serenity even at home, being able to avoid distractions and the ability to customize the difficulty depending on the individual skills.\\
The first meetings for the collection of the requisites were carried out as follows
\begin{enumerate}
\item First meeting: 
\begin{itemize}
\item[-] Questions:
\begin{itemize}
\item[*]Where is the feeding laboratory performed?
\item[*]How is the feeding laboratory performed?
\item[*]What types of topics are treated?
\item[*]What materials are used?
\item[*]What level of difficulty is reached?
\item[*]What are the most difficult issues?
\item[*]What is the relationship of young people with new technologies (ex.
VR viewer)?
\item[*]What should we try to avoid or limit?
\item[*]Can GEA be useful?
\item[*]How is the subdivision into three mini-games considered?
\end{itemize} 
\item[-]Participants: Therapists and patients, suffering from NDD syndrome, from \textit{Fraternit\'a \& Amicizia Onlus}
\item[-]Context: 8/11/2017 in a classroom of \textit{Fraternit\'a \& Amicizia Onlus}
\item[-]Execution: The questions were addressed to therapists respectively and guys, who have worked actively with a lot enthusiasm, and later the idea of GEA was showed and expressed opinions about that
\item[-]Result: The idea was met with great enthusiasm so we decide to continue
\end{itemize}
\item Second meeting:
\begin{itemize}
\item[-]Questions: Ask for an opinion regarding graphics, setting, content and structure of games
\item[-]Participants: Eleonora Beccaluva, therapist of the center \textit{Fraternit\'a \& Amicizia Onlus}
\item[-]Context: 23/11/2017 in a classroom of the I3Lab laboratory at the Milan Polytechnic
\item[-]Execution: The mockups were shown to Eleonora Beccaluva and the questions were asked
\item[-]Result: We were provided with suggestions regarding graphics and underlined the fact that not all children are able to read
\end{itemize}
\end{enumerate}
\subsection{Onlus varie}
\subsection{Main target groups}
Our system is designed to be easily used by different kinds of users that must have a little bit knowledge about nutrition, like for example the food pyramid. There are three main categories of stakeholders involved in our application:
\begin{itemize}
\item The first group is composed by children with NDD because they can have great benefits in using it. The application doesn't have a target age for this group, but it is important to note that during our conversations with therapists, it emerged that WIVR experiences and social experiences are usually proposed to people with mild to low NDDs. It can also be used by children not affected by this syndrome but it can result "simple".
\item In the second group there are therapists, in hospitals or organizations, educators and all the other people that have to teach food education to NDD children. They can integrate their lessons with a game session using GEA to improve understanding and have a feedback on children's knowledge so it can be used in specialized centers or schools but also at home because you need only few technological instruments and not very high capabilities in computer science.
\item The third group encloses developers, managers, researchers, designers, VR companies and all the people that can be affected by GEA diffusion and results.
\end{itemize} 
\subsection{Context and need addressed}
The context of use of GEA is predominantly a room in a specialized center during
a feeding laboratory session with the presence of a therapist. It can also be used at home, even if preferably with the presence of someone who can explain the mistakes made. Finally it could be used in schools for post-lesson learning exercises and tests.
\subsection{Constraints}
\subsection{Goals}
\subsection{Requirements}
elencati

\section{UX Design}
\section{Implementation Overview}