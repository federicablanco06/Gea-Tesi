\section{Requirements elicitation}
\subsection{Onlus varie}
\subsection{Main target groups}
There are three main categories of stakeholders involved in our application:
\begin{itemize}
\item The first group is composed by children with NDD because they can have great benefits in using it. It can also be used by children not affected by this syndrome but it can result "simple".
\item In the second group there are therapists, in hospitals or organizations, educators and all the other people that have to teach food education to NDD children. They can integrate their lessons with a game session using GEA to improve understanding and have a feedback on children's knowledge so it can be used in specialized centers or schools but also at home because you need only few technological instruments and not very high capabilities in computer science.
\item The third group encloses developers, managers, researchers, designers, VR companies and all the people that can be affected by GEA diffusion and results.
\end{itemize} 
\subsection{Context and need addressed}
\subsection{Constraints}
\subsection{Goals}
\subsection{Requirements}
elencati

\section{UX Design}
\section{Implementation Overview}